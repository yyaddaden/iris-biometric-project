\block{Méthodologie d'évaluation}{
 	{ \fontsize{40}{40}\selectfont 
 	
	\begin{itemize}
		\item[\amark] La base de données de référence \href{https://www.kaggle.com/datasets/naureenmohammad/mmu-iris-dataset}{\textbf{MMU iris dataset}} a été téléchargée à partir de \textbf{Kaggle}.
		
		\item[\amark] Caractéristiques biométriques pour $\mathbf{46}$ personnes, mais uniquement $\mathbf{10}$ sélectionnées aléatoirement ont été utilisées.
		
		\item[\amark] Durant les évaluations, c'est la \textit{validation croisée} qui a été adoptée. Elle permet de générer $\mathbf{K = 10}$ parties \textit{stratifiés} afin d'éviter les biais.
	\end{itemize}	 	
	}
}

\block{Résultats préliminaires}{
 	{ \fontsize{40}{40}\selectfont 
 	
 	
 	    \begin{center}
			\captionof{table}{\selectfont Matrice de confusion pour \textbf{GLCM}-\textbf{PCA} et $k$-\textbf{NN}. }
			\vspace{0.3em}
			{ \fontsize{25}{25}\selectfont  	
				\begin{tabular}{l|c|c|c|c|c|c|c|c|c|c}
				Sujets & \textbf{005} & \textbf{010} & \textbf{013} & \textbf{022} & \textbf{027} & \textbf{030} & \textbf{032} & 						\textbf{035} & \textbf{038} & \textbf{040} \\ \hline
				\textbf{005} & \textbf{6} & 0 & 0 & 0 & 0 & 0 & 1 & 0 & 0 & 3 \\ \hline
				\textbf{010} & 0 & \textbf{9} & 0 & 1 & 0 & 0 & 0 & 0 & 0 & 0 \\ \hline
				\textbf{013} & 0 & 0 & \textbf{8} & 0 & 0 & 1 & 0 & 0 & 0 & 1 \\ \hline
				\textbf{022} & 0 & 1 & 0 & \textbf{8} & 1 & 0 & 0 & 0 & 0 & 0 \\ \hline
				\textbf{027} & 0 & 1 & 0 & 2 & \textbf{6} & 0 & 1 & 0 & 0 & 0 \\ \hline
				\textbf{030} & 2 & 0 & 1 & 0 & 0 & \textbf{4} & 1 & 0 & 0 & 2 \\ \hline
				\textbf{032} & 0 & 1 & 0 & 0 & 1 & 0 & \textbf{7} & 0 & 0 & 1 \\ \hline
				\textbf{035} & 0 & 1 & 0 & 0 & 1 & 1 & 0 & \textbf{7} & 0 & 0 \\ \hline
				\textbf{038} & 0 & 0 & 0 & 0 & 1 & 0 & 0 & 0 & \textbf{9} & 0 \\ \hline
				\textbf{040} & 0 & 1 & 1 & 0 & 0 & 1 & 0 & 0 & 0 & \textbf{7} \\ \hline
				\end{tabular}
			}
		\end{center}
    
    \begin{center}
		\captionof{table}{\selectfont Comparaison des deux méthodes.}
		\vspace{0.3em}
		{ \fontsize{30}{30}\selectfont 
    	\begin{tabular}{l|c}
        	Méthode 	& Taux de reconnaissance \\ \hline
            \textbf{GLCM}-\textbf{PCA} et $k$-\textbf{NN}   & $\mathbf{71\%}$ \\ \hline
            \textbf{GLCM} et $k$-\textbf{NN}     			& $22\%$ \\ \hline
    	\end{tabular}     		
    	}
    \end{center}
    \vspace{1em}
    
    }
}

\block{Conclusion}{
 	{ \fontsize{40}{40}\selectfont 
 	
	\begin{itemize}
		\item[\cmark] Obtention de résultats intéressants avec la représentation \textbf{GLCM}-\textbf{PCA} qui est bien meilleure que la simple \textbf{GLCM} (+ \textit{caractéristiques}). 

		\item[\cmark] Améliorer le processus de pré-traitement pour mieux détecter l'iris dans l'image et pour retirer le bruit comme les cils.
	\end{itemize}
    
    }
}