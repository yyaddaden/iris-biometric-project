\block{Évaluation}{
 	{ \fontsize{40}{40}\selectfont 
 	
 	La base de donéees utilisée pour ce projet est « MMU iris dataset ». Elle est composée d’images d’oeil pour l’entraînement de modèles de système biométrique basé sur l’iris de l’oeil. Cet ensemble de données se compose de 5 images de l’iris gauche et droit de 46 personnes.\cite{ref_2} Les tests ont été faits avec un échantillon aléatoire de 10 personnes dans la base de données.
\\~\\
	J'ai utilisé la validation croisée comme stratégie d’évaluation. Elle consiste à séparer les données en, par exemple, 10 sous-groupes aléatoires pour l’entraînement. Ensuite, on entraîne et évalue les données 10 fois en choisissant 9 des sous-groupes pour l’entraînement et le dernier pour l’évaluation.\cite{ref_1} J'ai utilisé un k-fold = 10.
	
	}
}

\block{Résultats}{
 	{ \fontsize{40}{40}\selectfont 
 	
    Deux méthodes ont été testées pour identifier celle qui donne de meilleurs résultats.
    
	\vspace{1em}
    \begin{center}
		\captionof{table}{\selectfont Comparaison des méthodes. }
		{ \fontsize{25}{25}\selectfont 
    	\begin{tabular}{lcc}
        	Méthode 	& Précision \\ \hline
            GLCM-PCA-KNN     	& 71\% \\ \hline
            GLCM-KNN     	& 22\% \\ \hline
    	\end{tabular}     		
    	}
    \end{center}
    \vspace{1em}
    
    }
}

\block{Conclusion}{
 	{ \fontsize{40}{40}\selectfont 
 	
    Après avoir testé les deux méthodes, on a pu conclure que le méthode GLCM-PCA-KNN est meilleure pour identifier les personnes à partir de l'iris. Pour obtenir de meilleurs résultats, on aurait pu améliorer le processus de pré-traitement pour mieux détecter l'iris dans l'image et pour retirer le bruit comme les cils.
    
    }
}